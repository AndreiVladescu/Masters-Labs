MTD can best be described as the classical game of "shell game" \cite{cho2019proactive}, in which a stone is hidden under three cups or shells, and the gambler tries to guess the location of the stone, after they were shuffled.\\
In the words of \cite{mtd_vol1}, moving target defenses have been proposed as a way to make it much more difficult for an attacker to exploit a vulnerable system by changing aspects of that system to present attackers with a varying attack surface. The goal of a diverse defense is to make the attack target unpredictable, making it more difficult to deliver a malicious packet. Such strategies are already in place in a computer, such as address space layout randomization (ASLR), instruction set randomization (ISR), honeypots (decoy nodes) or honeynets (decoy networks). As such, different parameters can be varied so that the attacker will either miss his target, or worse, hit a decoy target.\\
In the space of IoT, the targets are usually small nodes, with a low power consumption, sometimes in hard-to-reach locations. Some of these attacks, that aim to bring down nodes will catastrophically affect these devices, as they may be either something trivial as a light bulb, or something that belongs to a city's infrastructure \cite{iot_electric_grid}. These MTD methodologies must be effective, in relation to a certain attack, must not add much overhead, as IoT devices are computationally slow, and they will possibly consume more power only by adding a framework specifically for this defense mechanism.